\documentclass[11pt]{article}

\usepackage[dvips]{graphicx}
\usepackage[rflt]{floatflt}
\usepackage{amsmath,amsthm,amssymb}

\renewcommand{\baselinestretch}{1.0}
\parindent0em
\parskip1.5ex plus0.5ex minus0.5ex
\topmargin-0.8cm
\oddsidemargin0.2cm
\textwidth16cm
\textheight22cm
\setlength{\unitlength}{1cm}
%
\begin{document}
\thispagestyle{empty}


\centerline{\bf PHY 482, HW\#4}

\begin{itemize}
%
\item[1.] Superposition of stationary states\\
The normalized state of a system is given by a superposition of stationary states as:
\begin{equation}
\Psi(x,t) = \sum\limits_{n=1}^{\infty} c_n \chi_n(x) \exp(-iE_n t)
\end{equation}
\begin{itemize}
\item[a)]
Show that from the normalization of $\Psi(x,t)$ follows that
\begin{equation}
\sum\limits_{n=1}^{\infty} |c_n|^2 = 1
\end{equation}
Give a physical interpretation of Eq.\ (2).
\item[b)]
Solve Eq.\ (1) for $c_5$.
\end{itemize}
%
\vspace*{0.25cm}
%
\item[2.] Energy measurement I\\
A system is initially in the state
\begin{displaymath}
\Psi(x,t=0) =
\frac{1}{\sqrt{7}}\left(
\sqrt{2}\chi_1(x)
+\sqrt{3}\chi_2(x)
-i\chi_3(x)
+\chi_4(x)
\right)
\end{displaymath}
where $\chi_n(x)$ are eigenstates of the system's Hamiltonian such that ${\hat H}\chi_n(x)=n^2 E_0\chi_n(x)$
\begin{itemize}
\item[a)]
Is $\Psi(x,t)$ normalized for $t>0$?
\item[b)]
If the energy of state is measured, what values will be obtained and with what probabilities?
\item[c)]
Calculate the expectation value of the total energy of the system.
\item[d)]
Suppose that a measurement yields $4E_0$, write down the wave function immediately after the measurement.
\end{itemize}
%
\vspace*{0.25cm}
%
\item[3.] Energy measurement II\\
Consider a particle in an (1D) infinite square well of width $a$. The normalized energy eigenstates of this system are given by \begin{displaymath}
\Psi_n(x,t=0) = \sqrt{\frac{2}{a}} \sin\left(\frac{n\pi x}{a}\right)
\end{displaymath}
and the corresponding energy eigenvalues are given by $E_n =\frac{n^2\pi^2\hbar^2}{2ma^2}$.
\\ \\
%
\centerline{- Problem continues on the back -}
\newpage
%
Suppose that the wave function of the particle at time $t=0$ is given by
\begin{displaymath}
\Psi(x,t=0) =
\begin{cases}
A\frac{x}{a}, &\mbox{if}\quad 0\le x\le \frac{a}{2}\\
A\left(1-\frac{x}{a}\right), &\mbox{if}\quad \frac{a}{2} < x \le a\\
0, &\mbox{else}
\end{cases}
\end{displaymath}
where $A = \sqrt{\frac{12}{a}}$ such that $\int\limits_{-\infty}^{\infty} |\Psi(x,t=0)|^2 dx = 1$. Calculate the probability that the measurement of the energy yields the eigenvalue $E_3$.\\
{\it (Hint: Problem 1b) and its solution should be useful here.)}
%
\vspace*{0.25cm}
%
\item[4.] Parseval's theorem for Fourier transforms\\
For a function $f(x)$ its Fourier transform $F(k)$ is defined by
\begin{displaymath}
F(k) = \frac{1}{\sqrt{2\pi}}\int\limits_{-\infty}^{\infty}f(x)\exp(-ikx) dx
\end{displaymath}
while
\begin{displaymath}
f(x) = \frac{1}{\sqrt{2\pi}}\int\limits_{-\infty}^{\infty}F(k)\exp(ikx) dk
\end{displaymath}
is called the inverse Fourier transform of $F(k)$.
Prove Parseval's theorem
\begin{displaymath}
\int\limits_{-\infty}^{\infty} |f(x)|^2 dx = \int\limits_{-\infty}^{\infty} |F(k)|^2 dk
\end{displaymath}
(This theorem tells you that if a function $f(x)$ is normalized, then its Fourier transform $F(k)$ is normalized too and vice versa.)
%
\vspace*{0.25cm}
%
\item[5.] Some physics with plane waves\\
Consider the plane wave characterized by a positive constant $k_0$
\begin{equation}\label{plane}
\Psi(x,t) = A \exp(ik_0x-i\hbar k_0^2t/2m)
\end{equation}
(Note: You do not have to determine $A$ for solving the following problems.)
\begin{itemize}
\item[a)]
Find the Fourier transform $\Phi(k,t)$ of this wavefunction. Describe in
words: What is the result telling you about plane waves?
\item[b)]
Find the probability current
\begin{displaymath}
J(x,t) = \frac{i\hbar}{2m}\left(\Psi\frac{\partial \Psi^*}{\partial x}
-\Psi^*\frac{\partial\Psi}{\partial x}\right)
\end{displaymath}
for the wavefunction given in Eq.\ (\ref{plane}).
Try to simplify your result as much as
possible and give a brief interpretation of the expression you get.
\item[c)] If you flip the sign of $k_0$, describe what has changed physically
and mathematically about the state. How is this reflected in the results
from part b)?
\end{itemize}
%
\end{itemize}
%
\end{document}

