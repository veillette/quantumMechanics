\documentclass[11pt]{article}

\usepackage[dvips]{graphicx}
\usepackage[rflt]{floatflt}
\usepackage{amsmath,amsthm,amssymb}

\renewcommand{\baselinestretch}{1.0}
\parindent0em
\parskip1.5ex plus0.5ex minus0.5ex
\topmargin-0.8cm
\oddsidemargin0.2cm
\textwidth16cm
\textheight22cm
\setlength{\unitlength}{1cm}
%
\begin{document}
\thispagestyle{empty}

\centerline{\bf PHY 482,  HW\#9}

\begin{itemize}
%
\item[1.] Schwarz inequality\\
Prove the Schwarz inequality
\begin{displaymath}
\langle A|A \rangle \langle B|B \rangle \ge |\langle A|B \rangle|^2
\end{displaymath}
(Hint: $Let |C> = |B> -\left(\frac{\langle A|B \rangle}{\langle A|A \rangle}\right) |A>$ and use the fact that for any
vector, i.e.\ in particular also for the vector $|C>$ yields $<C|C> \ge 0$.)
%
\vspace*{1cm}
%
\item[2.] Useful properties of commutators (and anti-commutators)
\begin{itemize}
\item[a)]
Assume ${\hat A}$ and ${\hat B}$ are Hermitian operators.
Show that the commutator $[{\hat A};{\hat B}] = {\hat A}{\hat B} - {\hat B}{\hat A}$ is anti-Hermitian, that means
$[{\hat A};{\hat B}] = -[{\hat A};{\hat B}]^{\dagger}$
and that the anti-commutator $\{{\hat A};{\hat B}\} = {\hat A}{\hat B} + {\hat B}{\hat A}$ is Hermitian.
\item[b)]
Show that $[{\hat x}^n; {\hat p}] = i\hbar n{\hat x}^{n-1}$.
\end{itemize}
%
\vspace*{1cm}
%
\item[3.] Uncertainty relation and time dependence of expectation values
\begin{itemize}
\item[a)]
Show that the generalized uncertainty principle for the operator ${\hat x}$ and ${\hat H}$ yields
\begin{displaymath}
\sigma_x \sigma_H \ge \frac{\hbar}{2m}|<p>|
\end{displaymath}
What can you deduce about the value of $<p>$ in a stationary state?
\item[b)]
Use
\begin{displaymath}
\frac{d}{dt}<{\hat A}> = \frac{1}{i\hbar} <[{\hat A},{\hat H}]> + <\frac{\partial {\hat A}}{\partial t}>
\end{displaymath}
with ${\hat A} = {\hat I}$ (identity operator), ${\hat A} = {\hat H}$ and ${\hat A} = {\hat p}$ to prove
the conservation of probability, the conservation of energy, and Ehrenfest's theorem, respectively.\\
(Hints: (i) In this class we did not and we will not consider problems, in which the potential is time-dependent. Thus, 
you can assume that $V(x,t) = V(x)$. (ii) In the last term of the right hand side ${\partial {\hat A}}/{\partial t}$ is the derivative of the {\it operator}, and only of the operator, with respect of time.)
\end{itemize}
%
\vspace*{1cm}
%
\centerline{\bf - Please note: There is another problem on the next page -}
\newpage
%
\item[4.] Matrix representation of the eigenvalue problem\\
Suppose there are two observables $A$ and $B$ with corresponding Hermitian operators ${\hat A}$ and ${\hat B}$. The eigenfunctions to ${\hat A}$ as well as to ${\hat B}$ form a complete set of basis functions, and satisfy the eigenvalue equations:
\begin{displaymath}
{\hat A} |a_n> = a_n |a_n> \qquad \mbox{and} \qquad {\hat B} |b_n> = b_n |b_n>\, .
\end{displaymath}
(Here we are adopting a common practice of labeling an eigenstate by its eigenvalues, so $|a_n>$ is the eigenvector of ${\hat A}$ with eigenvalue $a_n$. Do not confuse the eigenvalues, which are numbers, with their eigenvectors, which are vectors in Hilbert space).
\begin{itemize}
\item[a)]
Since the eigenfunctions form a complete set, any wavefunction $|\Psi(t)>$ can be written as an expansion in either basis (we assume for simplicity a discrete basis):
\begin{displaymath}
|\Psi (t)> = \sum\limits_{n} c_n(t) |a_n> = \sum\limits_{n} d_n(t) |b_n>
\end{displaymath}
Derive a formula that expresses any particular $d_n$ in terms of the set of $c_n$'s.
\item[b)]
Using the identity operator ${\hat I} = \sum\limits_i |b_i><b_i|$
show that ${\hat A}$ can be written as:
\begin{displaymath}
{\hat A} = \sum\limits_j\sum\limits_i A_{ji}|b_j><b_i|
\end{displaymath}
with $A_{ji} = <b_j|{\hat A}|b_i>$ is the $ji$-th element of the matrix representing ${\hat A}$.
\item[c)]
Show that the eigenvalue equation ${\hat A} |a_n> = a_n |a_n>$
can be written as
\begin{equation}
\sum\limits_i \left[A_{ji} - a_n \delta_{ij}\right] <b_i|a_n> = 0\, .
\end{equation}
\item[d)]
Equation (1) has nonzero solutions only if the determinant
\begin{displaymath}
\det(A - a_n I) = 0\, ,
\end{displaymath}
where $A$ is the matrix representing the operator ${\hat A}$ and $I$ is the identity matrix.
The solutions of this so-called secular or characteristic equation yield the eigenvalues $a_n$.\\
Use this result to find the eigenvalues $a_n$ of
\begin{displaymath}
A =
\begin{pmatrix}
7 & 0 & 0 \\
0 & 1 & -i \\
0 & i & -1
\end{pmatrix}
\end{displaymath}
Then, use the matrix representation of the eigenvalue equation to find the eigenvectors $|a_n>$ corresponding to the eigenvalues $a_n$. Make sure that the eigenvectors are normalized.
\end{itemize}
%
\end{itemize}
\end{document}

