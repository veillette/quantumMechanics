\documentclass[11pt]{article}

\usepackage[dvips]{graphicx}
\usepackage[rflt]{floatflt}
\usepackage{amsmath,amsthm,amssymb}

\renewcommand{\baselinestretch}{1.0}
\parindent0em
\parskip1.5ex plus0.5ex minus0.5ex
\topmargin-0.8cm
\oddsidemargin0.2cm
\textwidth16cm
\textheight22cm
\setlength{\unitlength}{1cm}
%
\begin{document}
\thispagestyle{empty}

\centerline{\bf PHY 482, HW\#8}

\begin{itemize}
%
\item[1.] Vectors and vector spaces
\begin{itemize}
\item[a)] 
    Consider ordinary 3D vectors $\vec{A} = A_x \hat{x} + A_y \hat{y} + A_z \hat{z}$ (with complex components) together with the set of scalars.  For each of the following three cases, find out whether it constitutes a vector space.  If so, what is its dimension?  If not, why not?
    \\ 
    (i) The subset of all vectors with $A_z = 0$.
    \\
    (ii) The subset of all vectors with $A_z = 1$.
    \\
    (iii) The subset of all vectors whose components are all equal.
\item[b)]
    Does the set of all $2 \times 2$ matrices form a vector space? Assume the usual rules for matrix addition and multiplication by a scalar:
    \begin{displaymath}
    \left( \begin{array}{cc}
        a   &   b\\
        c   &   d\\
    \end{array} \right)
    +
    \left( \begin{array}{cc}
        e   &   f\\
        g   &   h\\
    \end{array} \right)
    =
    \left( \begin{array}{cc}
        a+e   &   b+f\\
        c+g   &   d+h\\
    \end{array} \right)
    \quad
    \text{and}
    \quad
    \alpha
    \left( \begin{array}{cc}
        a   &   b\\
        c   &   d\\
    \end{array} \right)
    =
    \left( \begin{array}{cc}
        \alpha a   &   \alpha b\\
        \alpha c   &   \alpha d\\
    \end{array} \right)
    \end{displaymath}
    If it does not form a vector space, why not? If it does, state the dimensionality and give an example of a set of basis vectors.
\item[c)] 
    Does the set of all functions $f(x)$ defined on the range $0 < x < 1$ that vanish at $x=0$ and $x=1$ together with the set of scalars form a vector space?  If not, why not? If it does, state the dimensionality.
\end{itemize}
%
\vspace*{0.1cm}
%
\item[2.] Properties of Hermitian operators
\begin{itemize}
\item[a)] Show that the sum of two Hermitian operators is a Hermitian operator.
\item[b)] Suppose that $\hat{A}$ is a Hermitian operator and $\alpha$ is a number. Under what condition on $\alpha$ is $\alpha\hat{A}$ a Hermitian operator?
\item[c)] When is the product of two Hermitian operators a Hermitian operator?
\item[d)] Show that the Hamilton operator ${\hat H} = -\frac{\hbar^2}{2m}\frac{d^2}{dx^2} + V(x)$ is a Hermitian operator.
\end{itemize}
%
\vspace*{0.1cm}
\item[3.] Projection operators\\
An operator is said to be {\it idempotent}, if ${\hat P}^2 = {\hat P}$. If, in addition, ${\hat P}$ is an Hermitian operator, then ${\hat P}$ is called a projection operator.
\begin{itemize}
\item[a)] 
Show that $\left({\hat I} - {\hat P}\right)$ is a projection operator, if ${\hat P}$ is a projection operator (${\hat I}$ is the identity operator with ${\hat I} |\Psi\rangle = |\Psi\rangle$).
\item[b)]
Assume that a state vector $|\Psi \rangle$ can be written as 
\begin{displaymath}
|\Psi \rangle = |\Phi\rangle + |\chi\rangle
\end{displaymath}
with $|\Phi\rangle = {\hat P}|\Psi \rangle$ and $|\chi\rangle> = \left({\hat I} - {\hat P}\right)|\Psi \rangle$ where ${\hat P}$ is a projection operator and ${\hat I}$ is the identity operator. Show that $|\Phi\rangle$ and $|\chi\rangle$ are orthogonal.\\
(In other words, any state can be written in terms of two orthogonal states by means of a projection operator.) 
\end{itemize}
%
\centerline{\bf - There is another problem on the back -}
\newpage
%
\item[4.] Hermitian conjugate (or adjoint) of an operator\\
The Hermitian conjugate or Hermitian adjoint of an operator ${\hat A}$ is denoted as ${\hat A}^\dagger$ and 
is defined by (see Griffiths, Eq.\ 3.20)
\begin{displaymath}
\langle f | {\hat A}g \rangle = \langle {\hat A}^\dagger f | g\rangle
\end{displaymath}
\begin{itemize}
\item[a)]
In other books you find instead the following definition of the Hermitian conjugate:
\begin{displaymath}
\langle f | {\hat A}^\dagger g \rangle = \langle {\hat A} f | g\rangle
\end{displaymath}
Show that this definition is equivalent to the one above.
\item[b)]
Show that a Hermitian operator is self-adjoint, i.e.\ ${\hat A}^\dagger = {\hat A}$.
\item[c)]
Show the following properties of Hermitian conjugates of operators:\\
(i) $\left( {\hat A}^\dagger \right)^{\dagger} = {\hat A}$, 
(ii) $\left( \alpha {\hat A} \right)^\dagger = \alpha^* {\hat A}^\dagger$, and 
(iii) $\left({\hat A}{\hat B}\right)^\dagger = {\hat B}^\dagger {\hat A}^\dagger$
\item[d)]
Find the Hermitian conjugates of $i$, $\partial/\partial x$ and the raising operator ${\hat a}_+$.
\item[e)]
An operator ${\hat A}$ can be represented by a matrix $(A)_{ij}$. Show that the Hermitian conjugate of ${\hat A}$ is then represented by $(A^*)_{ji}$.
\end{itemize}
%
\end{itemize}
\end{document}

