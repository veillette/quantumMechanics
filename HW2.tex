\documentclass[11pt]{article}

\usepackage[dvips]{graphicx}
\usepackage[rflt]{floatflt}
\usepackage{amsmath,amsthm,amssymb}

\renewcommand{\baselinestretch}{1.0}
\parindent0em
\parskip1.5ex plus0.5ex minus0.5ex
\topmargin-0.8cm
\oddsidemargin0.2cm
\textwidth16cm
\textheight22cm
\setlength{\unitlength}{1cm}
%
\begin{document}
\thispagestyle{empty}


\centerline{\bf PHY 482,  HW\#2}

\begin{itemize}
%
\item[1.] A wavefunction
\begin{itemize}
%\item[a)]
%Using the known integral (you have used/calculated it in the previous HW set)
%\begin{equation}\label{int}
%\int_{-\infty}^{\infty}exp(-az^2)dz =\sqrt{\frac{\pi}{a}}
%\end{equation}
%find an expression for the integral $\int_{-\infty}^{\infty}z^2exp(-az^2)dz$ by differentiating equation (\ref{int})
%with respect to $a$.\\
%({\it Note: This is a very useful trick (not only) in quantum mechanics.})
\item[a)]
Consider the following wavefunction for a particle of mass $m$
\begin{equation}
\Psi(x,t) = A \left[\exp(-i\alpha\hbar t/m)+\beta x\exp(-3i\alpha\hbar t/m)\right]\exp(-\alpha x^2)
\end{equation}
where $\alpha$ and $\beta$ are real constants. Determine $A$ such that the wave function is normalized. Is the answer unique? Explain.\\
({\it Hint: You'll find the solutions to problem 3b) of the last HW set helpful here. Also, think about whether you can argue some terms are zero without explicitly calculating them.})
\item[b)]
What is the probability density for this wavefunction? Is the probability density independent of time?
\item[c)]
What is the expectation value $\langle x \rangle$? What kind of motion is the average value of the particle's position executing?
\end{itemize}
%
\item[2.] Thinking about the Schr\"odinger equation
\begin{itemize}
\item[a)]
Show that if $\Psi_1(x,t)$ and $\Psi_2(x,t)$ are solutions of the Schr\"odinger equation
\begin{equation}\label{schroed}
i\hbar \frac{\partial}{\partial t} \Psi(x,t) =
\left( -\frac{\hbar^2}{2m}\frac{\partial^2}{\partial x^2} + V(x,t) \right) \Psi(x,t)
\end{equation}
then the superposition $\Psi(x,t) = \Psi_1(x,t) + \Psi_2(x,t)$ is a solution too. 
\item[b)]
The composite Schr\"odinger equation
\begin{equation}\label{comp}
i\hbar \frac{\partial}{\partial t} \Psi(x_1,x_2,t) =
\left[
-\frac{\hbar^2}{2m}\left(\frac{\partial^2}{\partial x_1^2} + \frac{\partial}{\partial x_2^2}\right) + V_1(x_1,t) +V_2(x_2,t)
\right]
\Psi(x_1,x_2,t)
\end{equation}
describes a system of two independent particles.
Use the interpretation that $|\Psi(x_1,x_2,t)|^2$ is a probability density to show that the state of such a system must be given in the product form $\Psi(x_1,x_2,t) = \Psi_1(x_1,t)\Psi_2(x_2,t)$.\\[0.5cm]
%
\centerline{\bf - This problem continues on the back - }
%
\pagebreak
%
\item[c)]
Consider now two Schr\"odinger equations with different potential terms $V_1(x_1,t)$ and $V_2(x_2,t)$
\begin{eqnarray}\label{one}
i\hbar \frac{\partial}{\partial t} \Psi_1(x_1,t) =
\left( -\frac{\hbar^2}{2m}\frac{\partial^2}{\partial x_1^2} + V_1(x_1,t) \right) \Psi_1(x_1,t)
\\ \label{two}
i\hbar \frac{\partial}{\partial t} \Psi_2(x_2,t) =
\left( -\frac{\hbar^2}{2m}\frac{\partial^2}{\partial x_2^2} + V_2(x_2,t) \right) \Psi_2(x_2,t)
\end{eqnarray}
Show that if $\Psi_1(x_1,t)$ is a solution of equation $(\ref{one}$) and $\Psi_2(x_2,t)$ is a solution of equation $(\ref{two})$, then
the product form $\Psi(x_1,x_2,t) = \Psi_1(x_1,t)\Psi_2(x_2,t)$ is a solution of the Schr\"odinger equation
composed of the two equations above, namely
\begin{equation}\label{comp}
i\hbar \frac{\partial}{\partial t} \Psi(x_1,x_2,t) =
\left[
-\frac{\hbar^2}{2m}\left(\frac{\partial^2}{\partial x_1^2} + \frac{\partial}{\partial x_2^2}\right) + V_1(x_1,t) +V_2(x_2,t)
\right]
\Psi(x_1,x_2,t)
\end{equation}
({\it Note: This is the same equation as in part b).})
\item[d)]
Replace in part c) all time derivatives $\partial/\partial t$ by $\partial^2/\partial t^2$ and show that in this case
the product form is {\it not} solution of the composite equation.
\item[e)]
Review your answers and conclusions from parts b) to d) and argue why the time derivative in the Schr\"odinger equation should be $\partial/\partial t$ and {\it not} $\partial^2/\partial t^2$.
\end{itemize}
%
\item[3.] Phase factors\\
Let $\Psi(x,t)$ be a solution of the (one-dimensional) Schr\"odinger equation
\begin{equation}
i\hbar \frac{\partial}{\partial t}\Psi(x,t) =
\left[-\frac{\hbar^2}{2m}\frac{\partial^2}{\partial x^2}+V(x,t)\right]
\Psi(x,t).
\end{equation}
Suppose you add a constant $V_0$, which is independent of $x$ and $t$,
to the potential energy term $V(x,t)$.
\begin{itemize}
\item[a)] Show that the solution of the new Schr\"odinger equation is given by\\
$\Phi(x,t) = \Psi(x,t) \exp(-iV_0t/\hbar)$.
\item[b)] What effect does this time-dependent phase factor have on the
probability density $|\Phi(x,t)|^2$ and the expectation value of the position $\langle x \rangle$?
Explain.
\end{itemize}
%
\item[4.] Current density\\
In class we have shown that the continuity equation for probability is given by
\begin{equation}
\frac{d}{dt}|\Psi({\bf r},t)|^2 +{\bf \nabla} \cdot{\bf J}({\bf r},t) = 0.
\end{equation}
In deriving the equation we have assumed that the potential $V({\bf r},t)$ is a real
quantity.
\begin{itemize}
\item[a)]
Prove that if $V({\bf r},t)$ would be complex the continuity equation would become
\begin{equation}\label{imag}
\frac{d}{dt}|\Psi({\bf r},t)|^2 +{\bf \nabla} \cdot{\bf J}({\bf r},t) =
\frac{2}{\hbar} Im(V({\bf r},t)) |\Psi({\bf r},t)|^2
\end{equation}
\item[b)]
How do you interpret the term on the right hand side of equation (\ref{imag})? Discuss in particular
the cases $Im(V({\bf r},t))>0$ and  $Im(V({\bf r},t))<0$.
\end{itemize}
%
\end{itemize}
%
\end{document}

