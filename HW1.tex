\documentclass[11pt]{article}

\usepackage[dvips]{graphicx}
\usepackage[rflt]{floatflt}
\usepackage{amsmath,amsthm,amssymb}

\renewcommand{\baselinestretch}{1.0}
\parindent0em
\parskip1.5ex plus0.5ex minus0.5ex
\topmargin-0.8cm
\oddsidemargin0.2cm
\textwidth16cm
\textheight22cm
\setlength{\unitlength}{1cm}
%
\begin{document}
\thispagestyle{empty}


\centerline{\bf PHY 482,  HW\#1}

\begin{itemize}
%
\item[1.] Superposition of stationary states\\
The normalized state of a system is given by a superposition of stationary states as:
\begin{equation}
\Psi(x,t) = \sum\limits_{n=1}^{\infty} c_n \chi_n(x) \exp(-iE_nt)
\end{equation}
Show that from the normalization of $\Psi(x,t)$ follows that
\begin{equation}
\sum\limits_{n=1}^{\infty} |c_n|^2 = 1
\end{equation}
Give a physical interpretation of Eq.\ (2).
%
\vspace*{0.25cm}
%
\item[2.] Energy measurement I\\
A system is initially in the state
\begin{displaymath}
\Psi(x,t=0) =
\frac{1}{\sqrt{7}}\left(
\sqrt{2}\chi_1(x)
+\sqrt{3}\chi_2(x)
-i\chi_3(x)
+\chi_4(x)
\right)
\end{displaymath}
where $\chi_n(x)$ are eigenstates of the system's Hamiltonian such that ${\hat H}\chi_n(x)=n^2 E_0\chi_n(x)$
\begin{itemize}
\item[a)]
Is $\Psi(x,t)$ normalized for $t>0$?
\item[b)]
If the energy of state is measured, what values will be obtained and with what probabilities?
\item[c)]
Calculate the expectation value of the total energy of the system.
\item[d)]
Suppose that a measurement yields $4E_0$, write down the wave function immediately after the measurement.
\end{itemize}
%
\vspace*{0.25cm}
%
\item[3.] Energy measurement II\\
Consider a particle in an (1D) infinite square well of width $a$. The normalized energy eigenstates of this system are given by \begin{displaymath}
\Psi_n(x,t=0) = \sqrt{\frac{2}{a}} \sin\left(\frac{n\pi x}{a}\right)
\end{displaymath}
and the corresponding energy eigenvalues are given by $E_n =\frac{n^2\pi^2\hbar^2}{2ma^2}$.
\\ \\
%
\centerline{- Problem continues on the back -}
\newpage
%
Suppose that the wave function of the particle at time $t=0$ is given by
\begin{displaymath}
\Psi(x,t=0) =
\begin{cases}
A\frac{x}{a}, &\mbox{if}\quad 0\le x\le \frac{a}{2}\\
A\left(1-\frac{x}{a}\right), &\mbox{if}\quad \frac{a}{2} < x \le a\\
0, &\mbox{else}
\end{cases}
\end{displaymath}
where $A = \sqrt{\frac{12}{a}}$ such that $\int\limits_{-\infty}^{\infty} |\Psi(x,t=0)|^2 dx = 1$. Calculate the probability that the measurement of the energy yields the eigenvalue $E_3$.\\
{\it (Hint: Problem 1b) and its solution should be useful here.)}
%
\vspace*{0.25cm}
%

%
\item[4.] Some quantitative problems for a potential step
\begin{itemize}
%
\item[a)]
An electron with a kinetic energy of 10 eV at large negative values of $x$ is moving from left to right
along the $x$-axis. The potential energy is given by:
\begin{displaymath}
V(x) =
\begin{cases}
0 &x<0\\
20 \mbox{eV}, &x \ge 0
\end{cases}
\end{displaymath}
Let's define a penetration depth $\Delta x > 0$ such that $|\chi(x=\Delta x)|^2 = |\chi(x=0)|^2/e^2$. Calculate the penetration depth for the electron.
\item[b)]
Repeat part a) for a 70 kg person initially moving at 4 ms$^{-1}$ and running into a wall which can be represented
by a potential step of height equal to four times this person's kinetic energy before reaching the step.
\end{itemize}
%
\vspace*{0.5cm}
%
\item[5.] Scattering by a potential well\\
Consider the case of a particle beam which comes in from the left with $E>0$ and scatters of the potential
\begin{displaymath}
V(x) =
\begin{cases}
0, &x < 0\\
-V_0&0 < x < a\\
0, &a < x
\end{cases}
\end{displaymath}
with $V_0>0$ (note the minus sign in front of $V_0$ in the formula).
\begin{itemize}
\item[a)]
Write down the wave function in each region and give the wave numbers in your general solutions as functions of $E$ and $V_0$.
\item[b)]
Write down the appropriate boundary conditions using the wave functions you have determined and justify any terms in your general solutions you set to zero.
\item[c)]
The transmission coefficient for particles coming from the left to be transmitted to $x \rightarrow \infty$ is given by
\begin{displaymath}
T^{-1} = 1 + \frac{1}{4\frac{E}{V_0}\left(1+\frac{E}{V_0}\right)}\sin^2\left(\frac{a}{\hbar}\sqrt{2m(V_0+E)}\right)
\end{displaymath}
Determine $T$ for the limiting cases $E \rightarrow 0$ and $E \rightarrow \infty$.
\item[d)]
Consider the case where $E = V_0$. Determine $T$ at three different values of the parameters:
$V_0 \rightarrow 0$, $a^2mV_0/\hbar^2 = \pi^2/16$, $a^2mV_0/\hbar^2 = \pi^2/4$.
How does this compare to your classical expectation?
\end{itemize}
%
%
\item[6.] Scattering by a potential barrier\\
Consider now a potential with a barrier of height $V_0$
\begin{displaymath}
V(x) =
\begin{cases}
0, &x < 0\\
V_0&0 < x < a\\
0, &a < x
\end{cases}
\end{displaymath}
\begin{itemize}
\item[a)]
Adapt the results from the previous problem to demonstrate that for $0<E<V_0$ the transmission coefficient is given by
\begin{displaymath}
T^{-1} = 1+\frac{1}{4(E/V_0)(1-E/V_0)}\sinh^2\left(\frac{a}{\hbar}\sqrt{2m(V_0-E)}\right)
\end{displaymath}
To do this, turn the barrier downside up by replacing $V_0$ with $-V_0$ in that formula,
and obtain an expression for $T^{-1}$ where everything is real. It may help to know
that $\sinh(ix) = i \sin x$.
\item[b)]
Consider this system as a model of a baseball being thrown at a wall. A baseball has a
mass of about 150 g, and we take it to be thrown at 40 m/s (near 90 mph). Assume that the wall is
0.1 m thick, and let's make the approximation that the ball would have to be 5 times as energetic
to punch through the wall classically, so $V_0 = 5E$ with $E$ determined by the quantities above.\\
What is the order of magnitude for $T$? Let's put this into perspective: if you keep trying, tossing
a baseball at the wall once per second, roughly how long do you have to wait until it "pops through"
the wall quantum mechanically? Give your answer in seconds, and also in ages of the Universe
(current models show the Universe to be about 13.7 billion years old), and comment on your results.
\end{itemize}

\end{itemize}
\end{document}

