\documentclass[11pt]{article}

\usepackage[dvips]{graphicx}
\usepackage[rflt]{floatflt}
\usepackage{amsmath,amsthm,amssymb}

\renewcommand{\baselinestretch}{1.0}
\parindent0em
\parskip1.5ex plus0.5ex minus0.5ex
\topmargin-0.8cm
\oddsidemargin0.2cm
\textwidth16cm
\textheight22cm
\setlength{\unitlength}{1cm}
%
\begin{document}
\thispagestyle{empty}

\centerline{\bf PHY 482, Spring 14, HW\#7}

\begin{itemize}
%
\item[1.] Symmetric potentials (Griffiths, Problem 2.1(c))\\
Prove the following statement: If $V(x)$ is an even function (that is, $V(-x)=V(x)$), then the stationary state
wave functions $\chi(x)$ can always be taken to be either even or odd.
%
\vspace*{0.5cm}
%
%\item[2.] Ground state energy of infinite square well\\
%Calculate the uncertainty $\sigma_{p_n}=\sqrt{<p^2> - <p>^2}$ of the momentum of the particle in the $n$th state of the %infinite square well. Then use the result to estimate the ground state energy as $E_1 \approx \sigma_{p_1}^2/(2m)$. Compare %with the exact result.
%
%\vspace*{0.5cm}
%
\item[2.] Attractive delta potential\\
Consider a particle of mass $m$ subject to an attractive delta potential $V(x) = -V_0 \delta(x)$, where $V_0 > 0$.
Show that this particle has only one bound state. Find the energy and the wave function of the state.
%
\vspace*{0.5cm}
%
\item[3.] An expanding infinite square well\\
Consider a particle of mass $m$ in an infinite square well, that extends from 0 to $a$. Assume that the particle is in the ground state of the square well for $t<0$. At $t=0$ the size of the square well is suddenly expanded, so that it extends
from 0 to $2a$ leaving the wave function of the state undisturbed.
\begin{itemize}
\item[a)]
Write down the wave function $\Psi(x,t=0)$ in the new larger well (Think carefully about the different regions in the new well).
\item[b)]
The wave function $\Psi(x,t=0)$ can be expanded as a linear combination of the stationary states, $\chi_n(x)$, of the new larger well, i.e. $\Psi(x,t=0) = \sum\limits_{n=1}^\infty c_n \chi_n(x)$. Determine a formula for $c_n$ and show by summing the first terms that the sum over $|c_n|^2$ is approaching 1. Does this make sense? Explain.
\item[c)]
Assume that the energy of the particle is measured at some time $t > 0$. What is the expectation value $<E>$ of the energy? What is the probability that the energy of the first excited state ($n=2$) of the new well is measured?
\item[d)]
Calculate the smallest period of time, $\tau > 0$, at which $\Psi(x,\tau)$ = $\Psi(x,t=0)$.
\item[e)]
Draw a picture of $\Psi(x,t)$ at $t=\tau/2$.
\end{itemize}
%
\vspace*{0.5cm}
%
\centerline{\bf - There is another problem on the back - }
%
\newpage
%
\item[4.] Coherent states for the harmonic oscillator\\
In a harmonic oscillator a coherent state $\psi_\alpha(x)$ is defined as follows: When acted on by the lowering operator ${\hat a}_-$, we get the wave function back times a constant:
\begin{eqnarray}
\label{Eig}
{\hat a}_- \psi_\alpha (x) = \alpha \, \psi_\alpha(x) \,,
\end{eqnarray}
or in linear algebra language, $\psi_\alpha(x)$ is an eigenvector of ${\hat a}_-$ with eigenvalue $\alpha$.
Different coherent states have different values of $\alpha$.  Do {\em not} in general assume that the constant $\alpha$ is real.\\
({\em Coherent states have many applications in atomic, molecular, and optical physics.  For instance, lasers and Bose-Einstein condensates are examples of coherent states.})
%
\begin{itemize}
\item[a)]
Show that for any square-integrable functions $f(x)$ and $g(x)$
\begin{eqnarray}
\int\limits_{-\infty}^{\infty} f^*(x) ({\hat a}_{\pm} g(x)) dx
=  \int\limits_{-\infty}^{\infty} ({\hat a}_{\mp} f(x))^* g(x)) dx
\end{eqnarray}
\item[b)]
Use Eq.\ (2)
along with the eigenvector equation (\ref{Eig}) to evaluate $\langle x \rangle$ and $\langle p \rangle$ for the coherent state wave function $\psi_\alpha$ in terms of $\alpha$ and constants.  You may assume $\psi_\alpha(x)$ is normalized.
%OD
(Hint: How can $\hat{x}$ and $\hat{p}$ be written in terms of $a_+$ and $a_-$?)
\item[c)]
Is the ground state $\chi_0(x)$ of the harmonic oscillator a coherent state?  What is the value of $\alpha$?
\item[d)]
Any wave function of the harmonic oscillator can be expressed as a linear combination of
stationary states $\chi_n(x)$ of the harmonic oscillator. Assume therefore that
\begin{eqnarray}
\psi_\alpha(x) = \sum_{n=0}^\infty c_n \chi_n(x) \,,
\end{eqnarray}
and show that the $c_n$ are given by
\begin{eqnarray}
c_n = {\alpha^n \over \sqrt{n!}} \, c_0 \,.
\end{eqnarray}
(Hint: Operate with ${\hat a}_-$ on Eq.\ (3).)
\item[e)]
Another interesting property of coherent states is how their expectation values evolve in time.  Recall that stationary states have time-independent expectation values; coherent states are different.
Assume that $\Psi_\alpha(x, t=0) = \psi_\alpha(x)$ and show that $\Psi_\alpha(x,t)$ is still a coherent state --- that is, show it satisfies
\begin{eqnarray}
{\hat a}_- \Psi_\alpha (x,t ) = \alpha(t) \Psi_\alpha(x,t) \,.
\end{eqnarray}
What is $\alpha(t)$ in terms of $\alpha$ and other quantities?
\item[f)]
In this part, for simplicity assume $\alpha$ is real (but $\alpha(t)$ might not be real).  Take the results for $\langle x \rangle$ and $\langle p \rangle$ from part b) and put the value of $\alpha(t)$ into them to find $\langle x \rangle(t)$ and $\langle p \rangle (t)$  for $\Psi_\alpha(x,t)$.   How does the result compare to the classical motion of a particle in a harmonic oscillator?
\end{itemize}
\end{itemize}
\end{document}

